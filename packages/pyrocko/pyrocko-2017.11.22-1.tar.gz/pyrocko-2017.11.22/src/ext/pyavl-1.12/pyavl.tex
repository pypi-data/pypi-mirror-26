%&PDFLaTeX
\documentclass{howto}
\usepackage[T1]{fontenc}
\usepackage{fancyvrb}
\usepackage[]{hyperref}
\hypersetup{		
	a4paper,
	colorlinks,
	bookmarks=true,
	bookmarksopen=true,
	bookmarksnumbered=true,
	linkcolor=blue,
	urlcolor=blue,
	citecolor=red,
	pdfpagemode=UseOutlines,
	pdftitle={Python `avl' usage notes},
	pdfsubject={AVL-tree type for Python},
	pdfauthor={Richard McGraw},
	pdfcreator={pdfLaTeX}
}
\makeindex
% \makemodindex
\title{AVL-tree type for Python}
\author{R. McGraw}
\date{typeset \today}
\authoraddress{\email{dasnar@fastmail.fm}}
\release{1.1}
\begin{document}
\maketitle

% This makes the Abstract go on a separate page in the HTML version;
% if a copyright notice is used, it should go immediately after this.
%
\ifhtml
\chapter*{Front Matter\label{front}}
\fi

\begin{abstract}
\noindent This document describes how to use the \module{avl}
extension module (dynamically loadable library) for Python, which
implements a dual-personality object using AVL trees.  AVL trees
(named after the inventors, Adel'son-Vel'ski\u{\i} and Landis) are
balanced binary search trees.  While these objects can be seen as
implementing ordered containers, allowing fast lookup and deletion of
any item, they can also act as sequential lists.  The \module{avl}
module is based on a \C{} written library.  It is based on an
extension module by Sam Rushing (\cite{Rushing}) and on Ben Pfaff's
\module{libavl} (\cite{Pfaff}).
\end{abstract}

\tableofcontents

\section{\module{avl} ---
         AVL-tree type for Python}
% 
\declaremodule{extension}{avl}		% not standard, in C
\refexmodindex{avl}
% \platform{Unix}

\moduleauthor{author}{email}		% Author of the module code
\sectionauthor{Richard McGraw}{dasnar@fastmail.fm}		% Author of the documentation,
					% even if not a module section.
\modulesynopsis{AVL trees in Python}

The \module{avl} module defines dual-personality objects for Python
programmers to enjoy.  An object of type \code{`avl_tree'} can be seen
as implementing a dictionary, or it can act as a sequential list since
the underlying implementation maintains a \code{RANK} field at each
node in the tree.  Contrary to objects of type \code{`dict'} which
record key/value pairs, objects of type \code{`avl_tree'} record
single values, hereinafter referred to as `items'; it's possible to
insert duplicates.

%%%%%%%%%%%%%%%%%%%%%%%%
%                      %
%   Module functions   %
%                      %
%%%%%%%%%%%%%%%%%%%%%%%%
The \module{avl} module defines exactly these functions~:
%% 
 % Factory
 %%
\begin{funcdesc}{new}{%
	\optional{source\optional{, compare%
	\optional{, unique}}}%
	}
Create a new tree.  It is created empty if no argument is passed.  The
optional \var{source} argument can be either of type \code{`list'} or
\code{`avl_tree'}~: in the former case, each item from the list is
inserted into the tree; in the latter case, a copy of \var{object} is
returned.  Note that if \var{source} is a list which is known to be
sorted with respect to some \var{compare} function, it is more
efficient to call
\samp{avl.from_iter(iter(source),len(source),compare)}, see below.

The optional \var{compare} argument is a Python function that will be
used to order the tree instead of the default built-in mechanism.

If \var{\code{bool}==1} and source object is a list,
duplicates are removed (default: $0$). 
For example,
\begin{Verbatim}
>>> import avl
>>> a = avl.new([2,1,2], None, 1)
>>> type(a)
<type 'avl_tree'>
>>> a
[1, 2]
\end{Verbatim}
\end{funcdesc}

% 
% Pickling
% 
\begin{funcdesc}{dump}{tree,pickler}
	Convenience function to pickle a tree, as in
	\code{tree.dump(pickler)}.  First we pickle the size of the
	tree as a \code{PyInt} object, then its compare function, then
	each item in order by \code{pickler.dump()}.  The \module{cPickle}
	module has no exported API. The advantage of visiting the tree in
	inorder is that no comparison is necessary at unpickling time.
	
	Note that \var{pickler} is not type-checked (the function
	only checks for the existence of a callable `\var{dump}' attribute).
\end{funcdesc}
\begin{funcdesc}{load}{unpickler}
	Convenience function to unpickle a tree which was pickled with the
	simple method applied by \function{avl.dump} (see above).
	
	Note that \var{unpickler} is not type-checked (the
	function only checks for the existence of a callable `\var{load}'
	attribute).
\end{funcdesc}
\begin{funcdesc}{from_iter}{%
	iter\optional{, len},\optional{, compare}%
	}
	Load a tree from \var{iter} if the sequence is in sorted order
	with respect to some \var{compare} function.  This can't be
	reproduced in Python since it relies on \cfunction{avl_xload} in
	\code{avl.c} (which proceeds recursively like
	\cfunction{avl_slice} and has to know the items count in advance).
	If no \var{len} is specified, it will be read by the first call to
	\code{iter.next()}.  It can be specified as an \code{intobject} or
	a \code{longobject}.  If no \var{compare} function is specified,
	\code{None} is assumed.
	
	A \exception{StopIteration} exception occurs if \var{len} is too
	large.
	
	Note that \var{iter} is not type-checked (the function only checks
	for the existence of a callable `\var{next}' attribute).
\end{funcdesc}

\begin{excdesc}{avl.Error}
Exception raised when an operation fails for some \module{avl}
specific reason.  The exception argument is a string describing the
reason for failure or just where it occurred.
\end{excdesc}

%%%%%%%%%%%%%%%%%%%%%%%%
%                      %
%   AVL-tree objects   %
%                      %
%%%%%%%%%%%%%%%%%%%%%%%%
\subsection{`avl_tree' objects}
\label{avl-tree-objects}

\textsc{AVL} objects, as returned by \function{new()} above, have the
following functions~
\note{functions which do comparisons to do their work will raise
an exception as soon as some comparison procedure fails.}

First, here is a function that is included by defining the  
\constant{HAVE_AVL_VERIFY} compile flag~:
\begin{methoddesc}[avl_tree]{verify}{}
	Verify internal AVL tree structure, including \emph{ordering}.
	Return \code{1} if tree is valid, \code{0} otherwise.
\end{methoddesc}

\begin{methoddesc}[avl_tree]{lookup}{item}
	Return new reference to an item in the tree that compares equal to passed 
	\var{item}. Raise a \exception{LookupError} exception if \var{item} is
	not contained in the tree. 
\end{methoddesc}

\begin{methoddesc}[avl_tree]{insert}{item\optional{, index}}
	Insert \var{item} in the tree.  If no \var{index} is specified,
	\var{item} is inserted based on its rank with respect to the
	built-in Python compare function \cfunction{PyObject_Compare()},
	or a possible user-supplied one.  Indeed items of any type can be
	inserted as long as they are comparable to one another.  If some
	error occurs during a compare operation, the tree remains
	unchanged and an exception is raised.  Otherwise insertion is
	carried out whether or not an item comparing equal to \var{item}
	is already present in the tree.

	Use the call \samp{t.insert(o,j)} to insert \code{o} in front of
	index \code{j} in \code{t} regardless of order, if it's what you
	really want (an \exception{IndexError} exception is raised if
	\code{j} is out of range).  \note{order may be broken as a result. 
	See also the \method{span()} method.}
\end{methoddesc}

\begin{methoddesc}[avl_tree]{append}{item}
	Shortcut to append \var{item} to tree regardless of order.
\end{methoddesc}

\begin{methoddesc}[avl_tree]{remove}{item}
	Remove \var{item} from the tree. Nothing is done if \var{item} is not found.
\end{methoddesc}

\begin{methoddesc}[avl_tree]{clear}{}
	Make the tree logically empty in one sweep.
\end{methoddesc}

\begin{methoddesc}[avl_tree]{remove_at}{index}
	Remove item in front of specified \var{index}.
	An \code{IndexError} exception is raised if \var{index} is out of range.
\end{methoddesc}

\begin{methoddesc}[avl_tree]{has_key}{item}
	Return \code{1} if \var{item} is in tree, \code{0} otherwise.
\end{methoddesc}

\begin{methoddesc}[avl_tree]{index}{item}
	\samp{t.index(item)} returns smallest index \code{j} such that
	\code{t[j] == \var{item}}, or \code{-1} if \var{item} is not in \code{t}.
\end{methoddesc}

\begin{methoddesc}[avl_tree]{concat}{arg}
	In-place concatenation~: 
	append tree \var{arg} to the tree, regardless of order.
\end{methoddesc}

\begin{methoddesc}[avl_tree]{min}{}
	Return smallest item in tree, except if it's empty.
\end{methoddesc}

\begin{methoddesc}[avl_tree]{max}{}
	Return greatest item in tree, except if it's empty.
\end{methoddesc}

\begin{methoddesc}[avl_tree]{at_least}{item}
	Return smallest item that compares greater than or equal to \var{item} if 
	any, or raise a \exception{ValueError} exception.
\end{methoddesc}

\begin{methoddesc}[avl_tree]{at_most}{item}
	Return greatest item that compares less than or equal to \var{item} if 
	any, or raise a \exception{ValueError} exception.
\end{methoddesc}

% We might change that method's protocol
\begin{methoddesc}[avl_tree]{span}{o1\optional{, o2}}
	\samp{t.span(o1)} returns a pair of indices \code{(i,j)} such that 
	\code{t[i:j]} is the longest slice that spans \code{o1} if it's in \code{t},
	otherwise \code{i==j} in which \samp{t.insert(o1,j)} puts \code{o1} 
	where it should be with no need to redo comparisons.
	
	\samp{t.span(o1,o2)} returns \code{(i,j)} such that 
	\code{t[i:j]} is the longest slice that spans \code{o1,o2}.
	For example (see below for slice support),
\begin{Verbatim}
>>> a= avl.new(map (lambda x: random.randint(0,1000), range(10)))
>>> a
[28, 66, 82, 95, 109, 114, 268, 335, 761, 851]
>>> a.span(100,500)
(4, 8)
>>> a[4:8]
[109, 114, 268, 335]
>>> a.span(500,100)
(4, 8)
>>> a.span(900,950)
(10, 10)
>>> len(a)
10
>>> a.span(a[0],a[-1])
(0, 10)
>>> a.span(95,109)
(3, 5)
\end{Verbatim}
\end{methoddesc}

% Pickle method
\begin{methoddesc}[avl_tree]{dump}{pickler}
	See \function{avl.dump} module function.
\end{methoddesc}

% Get an iterator
\begin{methoddesc}[avl_tree]{iter}{\optional{pre_or_post}}
	Return a new iterator over the items in this tree, either in
	pre-position if \var{pre_or_post} is zero/false (the default), or in
	post-position.
\end{methoddesc}

%%%%%%%%%%%%%%%%%%%%%%%%%
%                       %
%   Sequence protocol   %
%                       %
%%%%%%%%%%%%%%%%%%%%%%%%%
There is support for the \textbf{sequence} protocol.

\begin{itemize}
	\item  \samp{len(t)} returns the size of tree \code{t}.

	\item  \samp{a+b} returns a new tree object resulting from the concatenation 
	of trees \code{`a'} and \code{`b'}. This is done regardless of order.

	\item  The \code{repeat} operation is undefined.
	
	\item  \code{`t[j]'} returns a new reference to the item whose inorder index 
	in \code{t} is \code{j} (starting from \code{0}).
	The usual defaults apply~: \code{j} should be in range 
	\code{[-len(t):len(t)]}, otherwise
	an \exception{IndexError} exception is raised.
	
	\item  A \strong{slice} of a tree can be obtained as a new tree, with the 
	usual defaults.
	Thus \samp{b = a[:]} is equivalent to \samp{b = avl.new(a)}.
	No exception.
	
	\item  Item and slice assignments are undefined.
	
	\item  \samp{o in t} is equivalent to \samp{t.has_key(o)}.
	
	\item To perform in-place concatenation please call \samp{a += b} 
	or \samp{a.concat(b)}.
\end{itemize}

%%%%%%%%%%%%%%%%%%%%%%%%%
%                       %
%   Iterator protocol   %
%                       %
%%%%%%%%%%%%%%%%%%%%%%%%%
\subsection{Support for the \textbf{iterator} protocol: `avl_tree_iterator' objects}
\label{iterator}

\begin{seealso}
\seepep{234}{Iterators}{New protocol in Python 2.2}
\end{seealso}

The \module{avl} module implements the iterator protocol. 
Objects of type \code{`avl_tree_iterator'} are implicitly called upon 
if a \code{for} loop is used to iterate over a tree,
like in \samp{for o in t: print o}. This is more 
efficient than retrieving the $i^{\mathrm{th}}$ item for all $i$.

There are also a couple of explicit methods.

To get a new iterator (in pre-position) for some tree \code{t}, say
\samp{iter(t)} in Python~:
\begin{Verbatim}
	>>> j = iter(t)
	>>> type(j)
	<type 'avl_tree_iterator'>
\end{Verbatim}
Note that it increments the tree object's refcount.

To obtain an iterator in either pre- or post-position,
use the \function{iter} instance method, see above.
The call \samp{t.iter()} is equivalent to \samp{iter(t)}.

Here are the methods~:

\begin{methoddesc}[avl_tree_iterator]{next}{}
	Return new reference to next item if there is one, or raise
	a  \code{StopIteration} exception. This is part of the protocol.
\end{methoddesc}

\begin{methoddesc}[avl_tree_iterator]{prev}{}
	Symmetrically, return new reference to previous item.
	This is not part of the protocol.
\end{methoddesc}

\begin{methoddesc}[avl_tree_iterator]{index}{}
	Return inorder index of current item in iteration, 
	or \code{-1} if it's in `pre-position', i.e. right before the
	first item, or tree's \code{len} if it's in `post-position'.
\end{methoddesc}

Since the tree implementation maintains a parent pointer for each
node, any iterator remains able to proceed if insertions are done
while it's in use, or if any item \emph{\textbf{other than the current
one}} is deleted via the \method{remove()} tree object method.  The
module provides another iterator method~:

\begin{methoddesc}[avl_tree_iterator]{remove}{}
	Remove current item in iteration, or raise an
	\exception{avl.Error} exception.  Current iterator position is set
	to next position or in post-position. This is not part of the protocol.
\end{methoddesc}

%%%%%%%%%%%%%%%
%             %
%   Example   %
%             %
%%%%%%%%%%%%%%%
\subsection{Example \label{avl-example}}

The following example demonstrates usage of the \module{avl} module.

\begin{Verbatim}[fontsize=\small,baselinestretch=.9,commentchar=!]
Python 2.3 (#1, Sep 13 2003, 00:49:11) 
[GCC 3.3 20030304 (Apple Computer, Inc. build 1495)] on darwin
Type "help", "copyright", "credits" or "license" for more information.
>>> import random
>>> import avl

# Create new avl_tree:
>>> t = avl.new()
>>> type(t)
<type 'avl_tree'>

# Insert new items:
>>> for x in range(20): t.insert(random.randint(0, 100))
... 
# verify() --> 1 if tree is valid, 0 otherwise
>>> t.verify()
1
>>> t
[2, 5, 9, 18, 24, 25, 29, 42, 45, 51, 58, 58, 60, 64, 66, 80, 85, 87, 92, 99]

# Lookup by index with usual syntax:
>>> t[0], t.min()
(2, 2)
>>> t[-1], t.max()
(99, 99)
>>> t[4], t[-16]
(24, 24)

# Create a tree from list of items:
>>> list = [3,8,3,5,1,2,8,7]
>>> u = avl.new(list)
# As a result, list is sorted ; this is an _unstable_ sort
>>> u
[1, 2, 3, 3, 5, 7, 8, 8]
# A second optional argument to new() is a Python compare function:
>>> u= avl.new(list, None)
>>> u
[1, 2, 3, 3, 5, 7, 8, 8]
# Same as above except duplicates are removed:
>>> u = avl.new(list, None, 1)
>>> u
[1, 2, 3, 5, 7, 8]

# Lookup by key:
>>> t.lookup(42)
42
# Failure:
>>> t.lookup(36)
Traceback (most recent call last):
  File "<stdin>", line 1, in ?
LookupError: 36

# Smallest index of item, or -1:
>>> t.index(58)
10
>>> i,j = t.span(58); i,j
(10, 12)
>>> t[10:12]
[58, 58]
>>> t.span(36)
(7, 7)
# Put it where it should be without comparing:
>>> t.insert(36,7); t.verify()
1
>>> t.index(36)
7

# Slices with usual defaults:
>>> u = t[:10]
>>> type(u)
<type 'avl_tree'>
>>> u
[2, 5, 9, 18, 24, 25, 29, 36, 42, 45]
>>> 42 in u
True
# otherwise put:
>>> u.has_key(42)
1

# new() accepts a tree as argument
# this is equivalent to `a = u[:]':
>>> a = avl.new(u)
>>> a.verify()
1
>>> b = avl.new(range(60,70))

### Concatenation with usual syntax:
>>> c = a + b
>>> c.verify()
1
>>> c
[2, 5, 9, 18, 24, 25, 29, 36, 42, 45, 60, 61, 62, 63, 64, 65, 66, 67, 68, 69]

# Concatenation is done regardless of order:
>>> avl.new([5,1,2]) + avl.new([2,8,6])
[1, 2, 5, 2, 6, 8]

# In-place concatenation:
>>> a.concat(b)
>>> a
[2, 5, 9, 18, 24, 25, 29, 36, 42, 45, 60, 61, 62, 63, 64, 65, 66, 67, 68, 69]

### ITERATION:
>>> n = 0
>>> for o in u:
... 	print n, ":", o
... 	n += 1
... 
0 : 2
1 : 5
2 : 9
3 : 18
4 : 24
5 : 25
6 : 29
7 : 36
8 : 42
9 : 45

### Explicit iteration:
>>> j = iter(u)
>>> type(j)
<type 'avl_tree_iterator'>
>>> j.next()
2
>>> j.next()
5
>>> u
[2, 5, 9, 18, 24, 25, 29, 36, 42, 45]
>>> j.index()
1
>>> u[2]
9
>>> u.insert(11)
>>> j.next()
9
>>> j.next()
11
>>> j.cur()
11
>>> j.remove()
>>> j.next()
18
>>> 11 in u
False
>>> u.verify()
1
# Get previous item:
>>> j.prev()
9

### Removing items:

# Remove u[3]:
>>> u.remove_at(3)
>>> u.verify()
1
# Remove by key:
>>> u.remove(24)
>>> u.remove(36)
>>> u
[2, 5, 9, 25, 29, 42, 45]
>>> u.verify()
1
# Nothing is done if item not found:
>>> u.remove(20)
>>> u
[2, 5, 9, 25, 29, 42, 45]

### at_least() and at_most(), with exceptions:
>>> u.at_least(30)
42
>>> u.at_most(30)
29
>>> u.at_least(50)
Traceback (most recent call last):
  File "<input>", line 1, in ?
ValueError: 50

! The span() protocol may be subject to change
>>> u.span(1,9)
(0, 3)
>>> u[0:3]
[2, 5, 9]
>>> u.span(2,8)
(0, 2)
>>> u.span(30,35)
(5, 5)
>>> u[5]
42
>>> u.span(25,44)
(3, 6)
>>> u[3:6]
[25, 29, 42]
>>> u.span(24,44)
(3, 6)
>>> u.span(24, 45)
(3, 7)
>>> u[3:7]
[25, 29, 42, 45]
>>> u.span(26,45)
(4, 7)
>>> 
\end{Verbatim}
% 
%%%%%%%%%%%%%%%%%%
%                %
%   References   %
%                %
%%%%%%%%%%%%%%%%%%
\clearpage
\begin{thebibliography}{9}\addcontentsline{toc}{section}{References}
	
	\bibitem{Rushing} Sam Rushing's own iterative C implementation, on
	which this one is based, is to be found at
	\url{http://www.python.org/ftp/python/contrib-09-Dec-1999/DataStructures/avl.README}
	
	\bibitem{Pfaff} Good ideas came from browsing 
	Ben Pfaff's GNU libavl home at \url{http://adtinfo.org/}
	
	\bibitem{AVL} Adel'son-Vel'ski\u{\i} (G.M.) and Landis (E.M.) ---
	{\itshape An algorithm for the organization of information.} Soviet
	Mathematics Doklady, vol.~3, 1962, pp.  1259--1263
\end{thebibliography}

%&PDFLaTeX
\documentclass{howto}
\usepackage[T1]{fontenc}
\usepackage{fancyvrb}
\usepackage[]{hyperref}
\hypersetup{		
	a4paper,
	colorlinks,
	bookmarks=true,
	bookmarksopen=true,
	bookmarksnumbered=true,
	linkcolor=blue,
	urlcolor=blue,
	citecolor=red,
	pdfpagemode=UseOutlines,
	pdftitle={Python `avl' usage notes},
	pdfsubject={AVL-tree type for Python},
	pdfauthor={Richard McGraw},
	pdfcreator={pdfLaTeX}
}
\makeindex
% \makemodindex
\title{AVL-tree type for Python}
\author{R. McGraw}
\date{typeset \today}
\authoraddress{\email{dasnar@fastmail.fm}}
\release{1.1}
\begin{document}
\maketitle

% This makes the Abstract go on a separate page in the HTML version;
% if a copyright notice is used, it should go immediately after this.
%
\ifhtml
\chapter*{Front Matter\label{front}}
\fi

\begin{abstract}
\noindent This document describes how to use the \module{avl}
extension module (dynamically loadable library) for Python, which
implements a dual-personality object using AVL trees.  AVL trees
(named after the inventors, Adel'son-Vel'ski\u{\i} and Landis) are
balanced binary search trees.  While these objects can be seen as
implementing ordered containers, allowing fast lookup and deletion of
any item, they can also act as sequential lists.  The \module{avl}
module is based on a \C{} written library.  It is based on an
extension module by Sam Rushing (\cite{Rushing}) and on Ben Pfaff's
\module{libavl} (\cite{Pfaff}).
\end{abstract}

\tableofcontents

\section{\module{avl} ---
         AVL-tree type for Python}
% 
\declaremodule{extension}{avl}		% not standard, in C
\refexmodindex{avl}
% \platform{Unix}

\moduleauthor{author}{email}		% Author of the module code
\sectionauthor{Richard McGraw}{dasnar@fastmail.fm}		% Author of the documentation,
					% even if not a module section.
\modulesynopsis{AVL trees in Python}

The \module{avl} module defines dual-personality objects for Python
programmers to enjoy.  An object of type \code{`avl_tree'} can be seen
as implementing a dictionary, or it can act as a sequential list since
the underlying implementation maintains a \code{RANK} field at each
node in the tree.  Contrary to objects of type \code{`dict'} which
record key/value pairs, objects of type \code{`avl_tree'} record
single values, hereinafter referred to as `items'; it's possible to
insert duplicates.

%%%%%%%%%%%%%%%%%%%%%%%%
%                      %
%   Module functions   %
%                      %
%%%%%%%%%%%%%%%%%%%%%%%%
The \module{avl} module defines exactly these functions~:
%% 
 % Factory
 %%
\begin{funcdesc}{new}{%
	\optional{source\optional{, compare%
	\optional{, unique}}}%
	}
Create a new tree.  It is created empty if no argument is passed.  The
optional \var{source} argument can be either of type \code{`list'} or
\code{`avl_tree'}~: in the former case, each item from the list is
inserted into the tree; in the latter case, a copy of \var{object} is
returned.  Note that if \var{source} is a list which is known to be
sorted with respect to some \var{compare} function, it is more
efficient to call
\samp{avl.from_iter(iter(source),len(source),compare)}, see below.

The optional \var{compare} argument is a Python function that will be
used to order the tree instead of the default built-in mechanism.

If \var{\code{bool}==1} and source object is a list,
duplicates are removed (default: $0$). 
For example,
\begin{Verbatim}
>>> import avl
>>> a = avl.new([2,1,2], None, 1)
>>> type(a)
<type 'avl_tree'>
>>> a
[1, 2]
\end{Verbatim}
\end{funcdesc}

% 
% Pickling
% 
\begin{funcdesc}{dump}{tree,pickler}
	Convenience function to pickle a tree, as in
	\code{tree.dump(pickler)}.  First we pickle the size of the
	tree as a \code{PyInt} object, then its compare function, then
	each item in order by \code{pickler.dump()}.  The \module{cPickle}
	module has no exported API. The advantage of visiting the tree in
	inorder is that no comparison is necessary at unpickling time.
	
	Note that \var{pickler} is not type-checked (the function
	only checks for the existence of a callable `\var{dump}' attribute).
\end{funcdesc}
\begin{funcdesc}{load}{unpickler}
	Convenience function to unpickle a tree which was pickled with the
	simple method applied by \function{avl.dump} (see above).
	
	Note that \var{unpickler} is not type-checked (the
	function only checks for the existence of a callable `\var{load}'
	attribute).
\end{funcdesc}
\begin{funcdesc}{from_iter}{%
	iter\optional{, len},\optional{, compare}%
	}
	Load a tree from \var{iter} if the sequence is in sorted order
	with respect to some \var{compare} function.  This can't be
	reproduced in Python since it relies on \cfunction{avl_xload} in
	\code{avl.c} (which proceeds recursively like
	\cfunction{avl_slice} and has to know the items count in advance).
	If no \var{len} is specified, it will be read by the first call to
	\code{iter.next()}.  It can be specified as an \code{intobject} or
	a \code{longobject}.  If no \var{compare} function is specified,
	\code{None} is assumed.
	
	A \exception{StopIteration} exception occurs if \var{len} is too
	large.
	
	Note that \var{iter} is not type-checked (the function only checks
	for the existence of a callable `\var{next}' attribute).
\end{funcdesc}

\begin{excdesc}{avl.Error}
Exception raised when an operation fails for some \module{avl}
specific reason.  The exception argument is a string describing the
reason for failure or just where it occurred.
\end{excdesc}

%%%%%%%%%%%%%%%%%%%%%%%%
%                      %
%   AVL-tree objects   %
%                      %
%%%%%%%%%%%%%%%%%%%%%%%%
\subsection{`avl_tree' objects}
\label{avl-tree-objects}

\textsc{AVL} objects, as returned by \function{new()} above, have the
following functions~
\note{functions which do comparisons to do their work will raise
an exception as soon as some comparison procedure fails.}

First, here is a function that is included by defining the  
\constant{HAVE_AVL_VERIFY} compile flag~:
\begin{methoddesc}[avl_tree]{verify}{}
	Verify internal AVL tree structure, including \emph{ordering}.
	Return \code{1} if tree is valid, \code{0} otherwise.
\end{methoddesc}

\begin{methoddesc}[avl_tree]{lookup}{item}
	Return new reference to an item in the tree that compares equal to passed 
	\var{item}. Raise a \exception{LookupError} exception if \var{item} is
	not contained in the tree. 
\end{methoddesc}

\begin{methoddesc}[avl_tree]{insert}{item\optional{, index}}
	Insert \var{item} in the tree.  If no \var{index} is specified,
	\var{item} is inserted based on its rank with respect to the
	built-in Python compare function \cfunction{PyObject_Compare()},
	or a possible user-supplied one.  Indeed items of any type can be
	inserted as long as they are comparable to one another.  If some
	error occurs during a compare operation, the tree remains
	unchanged and an exception is raised.  Otherwise insertion is
	carried out whether or not an item comparing equal to \var{item}
	is already present in the tree.

	Use the call \samp{t.insert(o,j)} to insert \code{o} in front of
	index \code{j} in \code{t} regardless of order, if it's what you
	really want (an \exception{IndexError} exception is raised if
	\code{j} is out of range).  \note{order may be broken as a result. 
	See also the \method{span()} method.}
\end{methoddesc}

\begin{methoddesc}[avl_tree]{append}{item}
	Shortcut to append \var{item} to tree regardless of order.
\end{methoddesc}

\begin{methoddesc}[avl_tree]{remove}{item}
	Remove \var{item} from the tree. Nothing is done if \var{item} is not found.
\end{methoddesc}

\begin{methoddesc}[avl_tree]{clear}{}
	Make the tree logically empty in one sweep.
\end{methoddesc}

\begin{methoddesc}[avl_tree]{remove_at}{index}
	Remove item in front of specified \var{index}.
	An \code{IndexError} exception is raised if \var{index} is out of range.
\end{methoddesc}

\begin{methoddesc}[avl_tree]{has_key}{item}
	Return \code{1} if \var{item} is in tree, \code{0} otherwise.
\end{methoddesc}

\begin{methoddesc}[avl_tree]{index}{item}
	\samp{t.index(item)} returns smallest index \code{j} such that
	\code{t[j] == \var{item}}, or \code{-1} if \var{item} is not in \code{t}.
\end{methoddesc}

\begin{methoddesc}[avl_tree]{concat}{arg}
	In-place concatenation~: 
	append tree \var{arg} to the tree, regardless of order.
\end{methoddesc}

\begin{methoddesc}[avl_tree]{min}{}
	Return smallest item in tree, except if it's empty.
\end{methoddesc}

\begin{methoddesc}[avl_tree]{max}{}
	Return greatest item in tree, except if it's empty.
\end{methoddesc}

\begin{methoddesc}[avl_tree]{at_least}{item}
	Return smallest item that compares greater than or equal to \var{item} if 
	any, or raise a \exception{ValueError} exception.
\end{methoddesc}

\begin{methoddesc}[avl_tree]{at_most}{item}
	Return greatest item that compares less than or equal to \var{item} if 
	any, or raise a \exception{ValueError} exception.
\end{methoddesc}

% We might change that method's protocol
\begin{methoddesc}[avl_tree]{span}{o1\optional{, o2}}
	\samp{t.span(o1)} returns a pair of indices \code{(i,j)} such that 
	\code{t[i:j]} is the longest slice that spans \code{o1} if it's in \code{t},
	otherwise \code{i==j} in which \samp{t.insert(o1,j)} puts \code{o1} 
	where it should be with no need to redo comparisons.
	
	\samp{t.span(o1,o2)} returns \code{(i,j)} such that 
	\code{t[i:j]} is the longest slice that spans \code{o1,o2}.
	For example (see below for slice support),
\begin{Verbatim}
>>> a= avl.new(map (lambda x: random.randint(0,1000), range(10)))
>>> a
[28, 66, 82, 95, 109, 114, 268, 335, 761, 851]
>>> a.span(100,500)
(4, 8)
>>> a[4:8]
[109, 114, 268, 335]
>>> a.span(500,100)
(4, 8)
>>> a.span(900,950)
(10, 10)
>>> len(a)
10
>>> a.span(a[0],a[-1])
(0, 10)
>>> a.span(95,109)
(3, 5)
\end{Verbatim}
\end{methoddesc}

% Pickle method
\begin{methoddesc}[avl_tree]{dump}{pickler}
	See \function{avl.dump} module function.
\end{methoddesc}

% Get an iterator
\begin{methoddesc}[avl_tree]{iter}{\optional{pre_or_post}}
	Return a new iterator over the items in this tree, either in
	pre-position if \var{pre_or_post} is zero/false (the default), or in
	post-position.
\end{methoddesc}

%%%%%%%%%%%%%%%%%%%%%%%%%
%                       %
%   Sequence protocol   %
%                       %
%%%%%%%%%%%%%%%%%%%%%%%%%
There is support for the \textbf{sequence} protocol.

\begin{itemize}
	\item  \samp{len(t)} returns the size of tree \code{t}.

	\item  \samp{a+b} returns a new tree object resulting from the concatenation 
	of trees \code{`a'} and \code{`b'}. This is done regardless of order.

	\item  The \code{repeat} operation is undefined.
	
	\item  \code{`t[j]'} returns a new reference to the item whose inorder index 
	in \code{t} is \code{j} (starting from \code{0}).
	The usual defaults apply~: \code{j} should be in range 
	\code{[-len(t):len(t)]}, otherwise
	an \exception{IndexError} exception is raised.
	
	\item  A \strong{slice} of a tree can be obtained as a new tree, with the 
	usual defaults.
	Thus \samp{b = a[:]} is equivalent to \samp{b = avl.new(a)}.
	No exception.
	
	\item  Item and slice assignments are undefined.
	
	\item  \samp{o in t} is equivalent to \samp{t.has_key(o)}.
	
	\item To perform in-place concatenation please call \samp{a += b} 
	or \samp{a.concat(b)}.
\end{itemize}

%%%%%%%%%%%%%%%%%%%%%%%%%
%                       %
%   Iterator protocol   %
%                       %
%%%%%%%%%%%%%%%%%%%%%%%%%
\subsection{Support for the \textbf{iterator} protocol: `avl_tree_iterator' objects}
\label{iterator}

\begin{seealso}
\seepep{234}{Iterators}{New protocol in Python 2.2}
\end{seealso}

The \module{avl} module implements the iterator protocol. 
Objects of type \code{`avl_tree_iterator'} are implicitly called upon 
if a \code{for} loop is used to iterate over a tree,
like in \samp{for o in t: print o}. This is more 
efficient than retrieving the $i^{\mathrm{th}}$ item for all $i$.

There are also a couple of explicit methods.

To get a new iterator (in pre-position) for some tree \code{t}, say
\samp{iter(t)} in Python~:
\begin{Verbatim}
	>>> j = iter(t)
	>>> type(j)
	<type 'avl_tree_iterator'>
\end{Verbatim}
Note that it increments the tree object's refcount.

To obtain an iterator in either pre- or post-position,
use the \function{iter} instance method, see above.
The call \samp{t.iter()} is equivalent to \samp{iter(t)}.

Here are the methods~:

\begin{methoddesc}[avl_tree_iterator]{next}{}
	Return new reference to next item if there is one, or raise
	a  \code{StopIteration} exception. This is part of the protocol.
\end{methoddesc}

\begin{methoddesc}[avl_tree_iterator]{prev}{}
	Symmetrically, return new reference to previous item.
	This is not part of the protocol.
\end{methoddesc}

\begin{methoddesc}[avl_tree_iterator]{index}{}
	Return inorder index of current item in iteration, 
	or \code{-1} if it's in `pre-position', i.e. right before the
	first item, or tree's \code{len} if it's in `post-position'.
\end{methoddesc}

Since the tree implementation maintains a parent pointer for each
node, any iterator remains able to proceed if insertions are done
while it's in use, or if any item \emph{\textbf{other than the current
one}} is deleted via the \method{remove()} tree object method.  The
module provides another iterator method~:

\begin{methoddesc}[avl_tree_iterator]{remove}{}
	Remove current item in iteration, or raise an
	\exception{avl.Error} exception.  Current iterator position is set
	to next position or in post-position. This is not part of the protocol.
\end{methoddesc}

%%%%%%%%%%%%%%%
%             %
%   Example   %
%             %
%%%%%%%%%%%%%%%
\subsection{Example \label{avl-example}}

The following example demonstrates usage of the \module{avl} module.

\begin{Verbatim}[fontsize=\small,baselinestretch=.9,commentchar=!]
Python 2.3 (#1, Sep 13 2003, 00:49:11) 
[GCC 3.3 20030304 (Apple Computer, Inc. build 1495)] on darwin
Type "help", "copyright", "credits" or "license" for more information.
>>> import random
>>> import avl

# Create new avl_tree:
>>> t = avl.new()
>>> type(t)
<type 'avl_tree'>

# Insert new items:
>>> for x in range(20): t.insert(random.randint(0, 100))
... 
# verify() --> 1 if tree is valid, 0 otherwise
>>> t.verify()
1
>>> t
[2, 5, 9, 18, 24, 25, 29, 42, 45, 51, 58, 58, 60, 64, 66, 80, 85, 87, 92, 99]

# Lookup by index with usual syntax:
>>> t[0], t.min()
(2, 2)
>>> t[-1], t.max()
(99, 99)
>>> t[4], t[-16]
(24, 24)

# Create a tree from list of items:
>>> list = [3,8,3,5,1,2,8,7]
>>> u = avl.new(list)
# As a result, list is sorted ; this is an _unstable_ sort
>>> u
[1, 2, 3, 3, 5, 7, 8, 8]
# A second optional argument to new() is a Python compare function:
>>> u= avl.new(list, None)
>>> u
[1, 2, 3, 3, 5, 7, 8, 8]
# Same as above except duplicates are removed:
>>> u = avl.new(list, None, 1)
>>> u
[1, 2, 3, 5, 7, 8]

# Lookup by key:
>>> t.lookup(42)
42
# Failure:
>>> t.lookup(36)
Traceback (most recent call last):
  File "<stdin>", line 1, in ?
LookupError: 36

# Smallest index of item, or -1:
>>> t.index(58)
10
>>> i,j = t.span(58); i,j
(10, 12)
>>> t[10:12]
[58, 58]
>>> t.span(36)
(7, 7)
# Put it where it should be without comparing:
>>> t.insert(36,7); t.verify()
1
>>> t.index(36)
7

# Slices with usual defaults:
>>> u = t[:10]
>>> type(u)
<type 'avl_tree'>
>>> u
[2, 5, 9, 18, 24, 25, 29, 36, 42, 45]
>>> 42 in u
True
# otherwise put:
>>> u.has_key(42)
1

# new() accepts a tree as argument
# this is equivalent to `a = u[:]':
>>> a = avl.new(u)
>>> a.verify()
1
>>> b = avl.new(range(60,70))

### Concatenation with usual syntax:
>>> c = a + b
>>> c.verify()
1
>>> c
[2, 5, 9, 18, 24, 25, 29, 36, 42, 45, 60, 61, 62, 63, 64, 65, 66, 67, 68, 69]

# Concatenation is done regardless of order:
>>> avl.new([5,1,2]) + avl.new([2,8,6])
[1, 2, 5, 2, 6, 8]

# In-place concatenation:
>>> a.concat(b)
>>> a
[2, 5, 9, 18, 24, 25, 29, 36, 42, 45, 60, 61, 62, 63, 64, 65, 66, 67, 68, 69]

### ITERATION:
>>> n = 0
>>> for o in u:
... 	print n, ":", o
... 	n += 1
... 
0 : 2
1 : 5
2 : 9
3 : 18
4 : 24
5 : 25
6 : 29
7 : 36
8 : 42
9 : 45

### Explicit iteration:
>>> j = iter(u)
>>> type(j)
<type 'avl_tree_iterator'>
>>> j.next()
2
>>> j.next()
5
>>> u
[2, 5, 9, 18, 24, 25, 29, 36, 42, 45]
>>> j.index()
1
>>> u[2]
9
>>> u.insert(11)
>>> j.next()
9
>>> j.next()
11
>>> j.cur()
11
>>> j.remove()
>>> j.next()
18
>>> 11 in u
False
>>> u.verify()
1
# Get previous item:
>>> j.prev()
9

### Removing items:

# Remove u[3]:
>>> u.remove_at(3)
>>> u.verify()
1
# Remove by key:
>>> u.remove(24)
>>> u.remove(36)
>>> u
[2, 5, 9, 25, 29, 42, 45]
>>> u.verify()
1
# Nothing is done if item not found:
>>> u.remove(20)
>>> u
[2, 5, 9, 25, 29, 42, 45]

### at_least() and at_most(), with exceptions:
>>> u.at_least(30)
42
>>> u.at_most(30)
29
>>> u.at_least(50)
Traceback (most recent call last):
  File "<input>", line 1, in ?
ValueError: 50

! The span() protocol may be subject to change
>>> u.span(1,9)
(0, 3)
>>> u[0:3]
[2, 5, 9]
>>> u.span(2,8)
(0, 2)
>>> u.span(30,35)
(5, 5)
>>> u[5]
42
>>> u.span(25,44)
(3, 6)
>>> u[3:6]
[25, 29, 42]
>>> u.span(24,44)
(3, 6)
>>> u.span(24, 45)
(3, 7)
>>> u[3:7]
[25, 29, 42, 45]
>>> u.span(26,45)
(4, 7)
>>> 
\end{Verbatim}
% 
%%%%%%%%%%%%%%%%%%
%                %
%   References   %
%                %
%%%%%%%%%%%%%%%%%%
\clearpage
\begin{thebibliography}{9}\addcontentsline{toc}{section}{References}
	
	\bibitem{Rushing} Sam Rushing's own iterative C implementation, on
	which this one is based, is to be found at
	\url{http://www.python.org/ftp/python/contrib-09-Dec-1999/DataStructures/avl.README}
	
	\bibitem{Pfaff} Good ideas came from browsing 
	Ben Pfaff's GNU libavl home at \url{http://adtinfo.org/}
	
	\bibitem{AVL} Adel'son-Vel'ski\u{\i} (G.M.) and Landis (E.M.) ---
	{\itshape An algorithm for the organization of information.} Soviet
	Mathematics Doklady, vol.~3, 1962, pp.  1259--1263
\end{thebibliography}

%&PDFLaTeX
\documentclass{howto}
\usepackage[T1]{fontenc}
\usepackage{fancyvrb}
\usepackage[]{hyperref}
\hypersetup{		
	a4paper,
	colorlinks,
	bookmarks=true,
	bookmarksopen=true,
	bookmarksnumbered=true,
	linkcolor=blue,
	urlcolor=blue,
	citecolor=red,
	pdfpagemode=UseOutlines,
	pdftitle={Python `avl' usage notes},
	pdfsubject={AVL-tree type for Python},
	pdfauthor={Richard McGraw},
	pdfcreator={pdfLaTeX}
}
\makeindex
% \makemodindex
\title{AVL-tree type for Python}
\author{R. McGraw}
\date{typeset \today}
\authoraddress{\email{dasnar@fastmail.fm}}
\release{1.1}
\begin{document}
\maketitle

% This makes the Abstract go on a separate page in the HTML version;
% if a copyright notice is used, it should go immediately after this.
%
\ifhtml
\chapter*{Front Matter\label{front}}
\fi

\begin{abstract}
\noindent This document describes how to use the \module{avl}
extension module (dynamically loadable library) for Python, which
implements a dual-personality object using AVL trees.  AVL trees
(named after the inventors, Adel'son-Vel'ski\u{\i} and Landis) are
balanced binary search trees.  While these objects can be seen as
implementing ordered containers, allowing fast lookup and deletion of
any item, they can also act as sequential lists.  The \module{avl}
module is based on a \C{} written library.  It is based on an
extension module by Sam Rushing (\cite{Rushing}) and on Ben Pfaff's
\module{libavl} (\cite{Pfaff}).
\end{abstract}

\tableofcontents

\section{\module{avl} ---
         AVL-tree type for Python}
% 
\declaremodule{extension}{avl}		% not standard, in C
\refexmodindex{avl}
% \platform{Unix}

\moduleauthor{author}{email}		% Author of the module code
\sectionauthor{Richard McGraw}{dasnar@fastmail.fm}		% Author of the documentation,
					% even if not a module section.
\modulesynopsis{AVL trees in Python}

The \module{avl} module defines dual-personality objects for Python
programmers to enjoy.  An object of type \code{`avl_tree'} can be seen
as implementing a dictionary, or it can act as a sequential list since
the underlying implementation maintains a \code{RANK} field at each
node in the tree.  Contrary to objects of type \code{`dict'} which
record key/value pairs, objects of type \code{`avl_tree'} record
single values, hereinafter referred to as `items'; it's possible to
insert duplicates.

%%%%%%%%%%%%%%%%%%%%%%%%
%                      %
%   Module functions   %
%                      %
%%%%%%%%%%%%%%%%%%%%%%%%
The \module{avl} module defines exactly these functions~:
%% 
 % Factory
 %%
\begin{funcdesc}{new}{%
	\optional{source\optional{, compare%
	\optional{, unique}}}%
	}
Create a new tree.  It is created empty if no argument is passed.  The
optional \var{source} argument can be either of type \code{`list'} or
\code{`avl_tree'}~: in the former case, each item from the list is
inserted into the tree; in the latter case, a copy of \var{object} is
returned.  Note that if \var{source} is a list which is known to be
sorted with respect to some \var{compare} function, it is more
efficient to call
\samp{avl.from_iter(iter(source),len(source),compare)}, see below.

The optional \var{compare} argument is a Python function that will be
used to order the tree instead of the default built-in mechanism.

If \var{\code{bool}==1} and source object is a list,
duplicates are removed (default: $0$). 
For example,
\begin{Verbatim}
>>> import avl
>>> a = avl.new([2,1,2], None, 1)
>>> type(a)
<type 'avl_tree'>
>>> a
[1, 2]
\end{Verbatim}
\end{funcdesc}

% 
% Pickling
% 
\begin{funcdesc}{dump}{tree,pickler}
	Convenience function to pickle a tree, as in
	\code{tree.dump(pickler)}.  First we pickle the size of the
	tree as a \code{PyInt} object, then its compare function, then
	each item in order by \code{pickler.dump()}.  The \module{cPickle}
	module has no exported API. The advantage of visiting the tree in
	inorder is that no comparison is necessary at unpickling time.
	
	Note that \var{pickler} is not type-checked (the function
	only checks for the existence of a callable `\var{dump}' attribute).
\end{funcdesc}
\begin{funcdesc}{load}{unpickler}
	Convenience function to unpickle a tree which was pickled with the
	simple method applied by \function{avl.dump} (see above).
	
	Note that \var{unpickler} is not type-checked (the
	function only checks for the existence of a callable `\var{load}'
	attribute).
\end{funcdesc}
\begin{funcdesc}{from_iter}{%
	iter\optional{, len},\optional{, compare}%
	}
	Load a tree from \var{iter} if the sequence is in sorted order
	with respect to some \var{compare} function.  This can't be
	reproduced in Python since it relies on \cfunction{avl_xload} in
	\code{avl.c} (which proceeds recursively like
	\cfunction{avl_slice} and has to know the items count in advance).
	If no \var{len} is specified, it will be read by the first call to
	\code{iter.next()}.  It can be specified as an \code{intobject} or
	a \code{longobject}.  If no \var{compare} function is specified,
	\code{None} is assumed.
	
	A \exception{StopIteration} exception occurs if \var{len} is too
	large.
	
	Note that \var{iter} is not type-checked (the function only checks
	for the existence of a callable `\var{next}' attribute).
\end{funcdesc}

\begin{excdesc}{avl.Error}
Exception raised when an operation fails for some \module{avl}
specific reason.  The exception argument is a string describing the
reason for failure or just where it occurred.
\end{excdesc}

%%%%%%%%%%%%%%%%%%%%%%%%
%                      %
%   AVL-tree objects   %
%                      %
%%%%%%%%%%%%%%%%%%%%%%%%
\subsection{`avl_tree' objects}
\label{avl-tree-objects}

\textsc{AVL} objects, as returned by \function{new()} above, have the
following functions~
\note{functions which do comparisons to do their work will raise
an exception as soon as some comparison procedure fails.}

First, here is a function that is included by defining the  
\constant{HAVE_AVL_VERIFY} compile flag~:
\begin{methoddesc}[avl_tree]{verify}{}
	Verify internal AVL tree structure, including \emph{ordering}.
	Return \code{1} if tree is valid, \code{0} otherwise.
\end{methoddesc}

\begin{methoddesc}[avl_tree]{lookup}{item}
	Return new reference to an item in the tree that compares equal to passed 
	\var{item}. Raise a \exception{LookupError} exception if \var{item} is
	not contained in the tree. 
\end{methoddesc}

\begin{methoddesc}[avl_tree]{insert}{item\optional{, index}}
	Insert \var{item} in the tree.  If no \var{index} is specified,
	\var{item} is inserted based on its rank with respect to the
	built-in Python compare function \cfunction{PyObject_Compare()},
	or a possible user-supplied one.  Indeed items of any type can be
	inserted as long as they are comparable to one another.  If some
	error occurs during a compare operation, the tree remains
	unchanged and an exception is raised.  Otherwise insertion is
	carried out whether or not an item comparing equal to \var{item}
	is already present in the tree.

	Use the call \samp{t.insert(o,j)} to insert \code{o} in front of
	index \code{j} in \code{t} regardless of order, if it's what you
	really want (an \exception{IndexError} exception is raised if
	\code{j} is out of range).  \note{order may be broken as a result. 
	See also the \method{span()} method.}
\end{methoddesc}

\begin{methoddesc}[avl_tree]{append}{item}
	Shortcut to append \var{item} to tree regardless of order.
\end{methoddesc}

\begin{methoddesc}[avl_tree]{remove}{item}
	Remove \var{item} from the tree. Nothing is done if \var{item} is not found.
\end{methoddesc}

\begin{methoddesc}[avl_tree]{clear}{}
	Make the tree logically empty in one sweep.
\end{methoddesc}

\begin{methoddesc}[avl_tree]{remove_at}{index}
	Remove item in front of specified \var{index}.
	An \code{IndexError} exception is raised if \var{index} is out of range.
\end{methoddesc}

\begin{methoddesc}[avl_tree]{has_key}{item}
	Return \code{1} if \var{item} is in tree, \code{0} otherwise.
\end{methoddesc}

\begin{methoddesc}[avl_tree]{index}{item}
	\samp{t.index(item)} returns smallest index \code{j} such that
	\code{t[j] == \var{item}}, or \code{-1} if \var{item} is not in \code{t}.
\end{methoddesc}

\begin{methoddesc}[avl_tree]{concat}{arg}
	In-place concatenation~: 
	append tree \var{arg} to the tree, regardless of order.
\end{methoddesc}

\begin{methoddesc}[avl_tree]{min}{}
	Return smallest item in tree, except if it's empty.
\end{methoddesc}

\begin{methoddesc}[avl_tree]{max}{}
	Return greatest item in tree, except if it's empty.
\end{methoddesc}

\begin{methoddesc}[avl_tree]{at_least}{item}
	Return smallest item that compares greater than or equal to \var{item} if 
	any, or raise a \exception{ValueError} exception.
\end{methoddesc}

\begin{methoddesc}[avl_tree]{at_most}{item}
	Return greatest item that compares less than or equal to \var{item} if 
	any, or raise a \exception{ValueError} exception.
\end{methoddesc}

% We might change that method's protocol
\begin{methoddesc}[avl_tree]{span}{o1\optional{, o2}}
	\samp{t.span(o1)} returns a pair of indices \code{(i,j)} such that 
	\code{t[i:j]} is the longest slice that spans \code{o1} if it's in \code{t},
	otherwise \code{i==j} in which \samp{t.insert(o1,j)} puts \code{o1} 
	where it should be with no need to redo comparisons.
	
	\samp{t.span(o1,o2)} returns \code{(i,j)} such that 
	\code{t[i:j]} is the longest slice that spans \code{o1,o2}.
	For example (see below for slice support),
\begin{Verbatim}
>>> a= avl.new(map (lambda x: random.randint(0,1000), range(10)))
>>> a
[28, 66, 82, 95, 109, 114, 268, 335, 761, 851]
>>> a.span(100,500)
(4, 8)
>>> a[4:8]
[109, 114, 268, 335]
>>> a.span(500,100)
(4, 8)
>>> a.span(900,950)
(10, 10)
>>> len(a)
10
>>> a.span(a[0],a[-1])
(0, 10)
>>> a.span(95,109)
(3, 5)
\end{Verbatim}
\end{methoddesc}

% Pickle method
\begin{methoddesc}[avl_tree]{dump}{pickler}
	See \function{avl.dump} module function.
\end{methoddesc}

% Get an iterator
\begin{methoddesc}[avl_tree]{iter}{\optional{pre_or_post}}
	Return a new iterator over the items in this tree, either in
	pre-position if \var{pre_or_post} is zero/false (the default), or in
	post-position.
\end{methoddesc}

%%%%%%%%%%%%%%%%%%%%%%%%%
%                       %
%   Sequence protocol   %
%                       %
%%%%%%%%%%%%%%%%%%%%%%%%%
There is support for the \textbf{sequence} protocol.

\begin{itemize}
	\item  \samp{len(t)} returns the size of tree \code{t}.

	\item  \samp{a+b} returns a new tree object resulting from the concatenation 
	of trees \code{`a'} and \code{`b'}. This is done regardless of order.

	\item  The \code{repeat} operation is undefined.
	
	\item  \code{`t[j]'} returns a new reference to the item whose inorder index 
	in \code{t} is \code{j} (starting from \code{0}).
	The usual defaults apply~: \code{j} should be in range 
	\code{[-len(t):len(t)]}, otherwise
	an \exception{IndexError} exception is raised.
	
	\item  A \strong{slice} of a tree can be obtained as a new tree, with the 
	usual defaults.
	Thus \samp{b = a[:]} is equivalent to \samp{b = avl.new(a)}.
	No exception.
	
	\item  Item and slice assignments are undefined.
	
	\item  \samp{o in t} is equivalent to \samp{t.has_key(o)}.
	
	\item To perform in-place concatenation please call \samp{a += b} 
	or \samp{a.concat(b)}.
\end{itemize}

%%%%%%%%%%%%%%%%%%%%%%%%%
%                       %
%   Iterator protocol   %
%                       %
%%%%%%%%%%%%%%%%%%%%%%%%%
\subsection{Support for the \textbf{iterator} protocol: `avl_tree_iterator' objects}
\label{iterator}

\begin{seealso}
\seepep{234}{Iterators}{New protocol in Python 2.2}
\end{seealso}

The \module{avl} module implements the iterator protocol. 
Objects of type \code{`avl_tree_iterator'} are implicitly called upon 
if a \code{for} loop is used to iterate over a tree,
like in \samp{for o in t: print o}. This is more 
efficient than retrieving the $i^{\mathrm{th}}$ item for all $i$.

There are also a couple of explicit methods.

To get a new iterator (in pre-position) for some tree \code{t}, say
\samp{iter(t)} in Python~:
\begin{Verbatim}
	>>> j = iter(t)
	>>> type(j)
	<type 'avl_tree_iterator'>
\end{Verbatim}
Note that it increments the tree object's refcount.

To obtain an iterator in either pre- or post-position,
use the \function{iter} instance method, see above.
The call \samp{t.iter()} is equivalent to \samp{iter(t)}.

Here are the methods~:

\begin{methoddesc}[avl_tree_iterator]{next}{}
	Return new reference to next item if there is one, or raise
	a  \code{StopIteration} exception. This is part of the protocol.
\end{methoddesc}

\begin{methoddesc}[avl_tree_iterator]{prev}{}
	Symmetrically, return new reference to previous item.
	This is not part of the protocol.
\end{methoddesc}

\begin{methoddesc}[avl_tree_iterator]{index}{}
	Return inorder index of current item in iteration, 
	or \code{-1} if it's in `pre-position', i.e. right before the
	first item, or tree's \code{len} if it's in `post-position'.
\end{methoddesc}

Since the tree implementation maintains a parent pointer for each
node, any iterator remains able to proceed if insertions are done
while it's in use, or if any item \emph{\textbf{other than the current
one}} is deleted via the \method{remove()} tree object method.  The
module provides another iterator method~:

\begin{methoddesc}[avl_tree_iterator]{remove}{}
	Remove current item in iteration, or raise an
	\exception{avl.Error} exception.  Current iterator position is set
	to next position or in post-position. This is not part of the protocol.
\end{methoddesc}

%%%%%%%%%%%%%%%
%             %
%   Example   %
%             %
%%%%%%%%%%%%%%%
\subsection{Example \label{avl-example}}

The following example demonstrates usage of the \module{avl} module.

\begin{Verbatim}[fontsize=\small,baselinestretch=.9,commentchar=!]
Python 2.3 (#1, Sep 13 2003, 00:49:11) 
[GCC 3.3 20030304 (Apple Computer, Inc. build 1495)] on darwin
Type "help", "copyright", "credits" or "license" for more information.
>>> import random
>>> import avl

# Create new avl_tree:
>>> t = avl.new()
>>> type(t)
<type 'avl_tree'>

# Insert new items:
>>> for x in range(20): t.insert(random.randint(0, 100))
... 
# verify() --> 1 if tree is valid, 0 otherwise
>>> t.verify()
1
>>> t
[2, 5, 9, 18, 24, 25, 29, 42, 45, 51, 58, 58, 60, 64, 66, 80, 85, 87, 92, 99]

# Lookup by index with usual syntax:
>>> t[0], t.min()
(2, 2)
>>> t[-1], t.max()
(99, 99)
>>> t[4], t[-16]
(24, 24)

# Create a tree from list of items:
>>> list = [3,8,3,5,1,2,8,7]
>>> u = avl.new(list)
# As a result, list is sorted ; this is an _unstable_ sort
>>> u
[1, 2, 3, 3, 5, 7, 8, 8]
# A second optional argument to new() is a Python compare function:
>>> u= avl.new(list, None)
>>> u
[1, 2, 3, 3, 5, 7, 8, 8]
# Same as above except duplicates are removed:
>>> u = avl.new(list, None, 1)
>>> u
[1, 2, 3, 5, 7, 8]

# Lookup by key:
>>> t.lookup(42)
42
# Failure:
>>> t.lookup(36)
Traceback (most recent call last):
  File "<stdin>", line 1, in ?
LookupError: 36

# Smallest index of item, or -1:
>>> t.index(58)
10
>>> i,j = t.span(58); i,j
(10, 12)
>>> t[10:12]
[58, 58]
>>> t.span(36)
(7, 7)
# Put it where it should be without comparing:
>>> t.insert(36,7); t.verify()
1
>>> t.index(36)
7

# Slices with usual defaults:
>>> u = t[:10]
>>> type(u)
<type 'avl_tree'>
>>> u
[2, 5, 9, 18, 24, 25, 29, 36, 42, 45]
>>> 42 in u
True
# otherwise put:
>>> u.has_key(42)
1

# new() accepts a tree as argument
# this is equivalent to `a = u[:]':
>>> a = avl.new(u)
>>> a.verify()
1
>>> b = avl.new(range(60,70))

### Concatenation with usual syntax:
>>> c = a + b
>>> c.verify()
1
>>> c
[2, 5, 9, 18, 24, 25, 29, 36, 42, 45, 60, 61, 62, 63, 64, 65, 66, 67, 68, 69]

# Concatenation is done regardless of order:
>>> avl.new([5,1,2]) + avl.new([2,8,6])
[1, 2, 5, 2, 6, 8]

# In-place concatenation:
>>> a.concat(b)
>>> a
[2, 5, 9, 18, 24, 25, 29, 36, 42, 45, 60, 61, 62, 63, 64, 65, 66, 67, 68, 69]

### ITERATION:
>>> n = 0
>>> for o in u:
... 	print n, ":", o
... 	n += 1
... 
0 : 2
1 : 5
2 : 9
3 : 18
4 : 24
5 : 25
6 : 29
7 : 36
8 : 42
9 : 45

### Explicit iteration:
>>> j = iter(u)
>>> type(j)
<type 'avl_tree_iterator'>
>>> j.next()
2
>>> j.next()
5
>>> u
[2, 5, 9, 18, 24, 25, 29, 36, 42, 45]
>>> j.index()
1
>>> u[2]
9
>>> u.insert(11)
>>> j.next()
9
>>> j.next()
11
>>> j.cur()
11
>>> j.remove()
>>> j.next()
18
>>> 11 in u
False
>>> u.verify()
1
# Get previous item:
>>> j.prev()
9

### Removing items:

# Remove u[3]:
>>> u.remove_at(3)
>>> u.verify()
1
# Remove by key:
>>> u.remove(24)
>>> u.remove(36)
>>> u
[2, 5, 9, 25, 29, 42, 45]
>>> u.verify()
1
# Nothing is done if item not found:
>>> u.remove(20)
>>> u
[2, 5, 9, 25, 29, 42, 45]

### at_least() and at_most(), with exceptions:
>>> u.at_least(30)
42
>>> u.at_most(30)
29
>>> u.at_least(50)
Traceback (most recent call last):
  File "<input>", line 1, in ?
ValueError: 50

! The span() protocol may be subject to change
>>> u.span(1,9)
(0, 3)
>>> u[0:3]
[2, 5, 9]
>>> u.span(2,8)
(0, 2)
>>> u.span(30,35)
(5, 5)
>>> u[5]
42
>>> u.span(25,44)
(3, 6)
>>> u[3:6]
[25, 29, 42]
>>> u.span(24,44)
(3, 6)
>>> u.span(24, 45)
(3, 7)
>>> u[3:7]
[25, 29, 42, 45]
>>> u.span(26,45)
(4, 7)
>>> 
\end{Verbatim}
% 
%%%%%%%%%%%%%%%%%%
%                %
%   References   %
%                %
%%%%%%%%%%%%%%%%%%
\clearpage
\begin{thebibliography}{9}\addcontentsline{toc}{section}{References}
	
	\bibitem{Rushing} Sam Rushing's own iterative C implementation, on
	which this one is based, is to be found at
	\url{http://www.python.org/ftp/python/contrib-09-Dec-1999/DataStructures/avl.README}
	
	\bibitem{Pfaff} Good ideas came from browsing 
	Ben Pfaff's GNU libavl home at \url{http://adtinfo.org/}
	
	\bibitem{AVL} Adel'son-Vel'ski\u{\i} (G.M.) and Landis (E.M.) ---
	{\itshape An algorithm for the organization of information.} Soviet
	Mathematics Doklady, vol.~3, 1962, pp.  1259--1263
\end{thebibliography}

%&PDFLaTeX
\documentclass{howto}
\usepackage[T1]{fontenc}
\usepackage{fancyvrb}
\usepackage[]{hyperref}
\input{pyavl-hysetup}
\makeindex
% \makemodindex
\title{AVL-tree type for Python}
\author{R. McGraw}
\date{typeset \today}
\authoraddress{\email{dasnar@fastmail.fm}}
\release{1.1}
\begin{document}
\maketitle

% This makes the Abstract go on a separate page in the HTML version;
% if a copyright notice is used, it should go immediately after this.
%
\ifhtml
\chapter*{Front Matter\label{front}}
\fi

\begin{abstract}
\noindent This document describes how to use the \module{avl}
extension module (dynamically loadable library) for Python, which
implements a dual-personality object using AVL trees.  AVL trees
(named after the inventors, Adel'son-Vel'ski\u{\i} and Landis) are
balanced binary search trees.  While these objects can be seen as
implementing ordered containers, allowing fast lookup and deletion of
any item, they can also act as sequential lists.  The \module{avl}
module is based on a \C{} written library.  It is based on an
extension module by Sam Rushing (\cite{Rushing}) and on Ben Pfaff's
\module{libavl} (\cite{Pfaff}).
\end{abstract}

\tableofcontents

\section{\module{avl} ---
         AVL-tree type for Python}
% 
\declaremodule{extension}{avl}		% not standard, in C
\refexmodindex{avl}
% \platform{Unix}

\moduleauthor{author}{email}		% Author of the module code
\sectionauthor{Richard McGraw}{dasnar@fastmail.fm}		% Author of the documentation,
					% even if not a module section.
\modulesynopsis{AVL trees in Python}

The \module{avl} module defines dual-personality objects for Python
programmers to enjoy.  An object of type \code{`avl_tree'} can be seen
as implementing a dictionary, or it can act as a sequential list since
the underlying implementation maintains a \code{RANK} field at each
node in the tree.  Contrary to objects of type \code{`dict'} which
record key/value pairs, objects of type \code{`avl_tree'} record
single values, hereinafter referred to as `items'; it's possible to
insert duplicates.

%%%%%%%%%%%%%%%%%%%%%%%%
%                      %
%   Module functions   %
%                      %
%%%%%%%%%%%%%%%%%%%%%%%%
The \module{avl} module defines exactly these functions~:
%% 
 % Factory
 %%
\begin{funcdesc}{new}{%
	\optional{source\optional{, compare%
	\optional{, unique}}}%
	}
Create a new tree.  It is created empty if no argument is passed.  The
optional \var{source} argument can be either of type \code{`list'} or
\code{`avl_tree'}~: in the former case, each item from the list is
inserted into the tree; in the latter case, a copy of \var{object} is
returned.  Note that if \var{source} is a list which is known to be
sorted with respect to some \var{compare} function, it is more
efficient to call
\samp{avl.from_iter(iter(source),len(source),compare)}, see below.

The optional \var{compare} argument is a Python function that will be
used to order the tree instead of the default built-in mechanism.

If \var{\code{bool}==1} and source object is a list,
duplicates are removed (default: $0$). 
For example,
\begin{Verbatim}
>>> import avl
>>> a = avl.new([2,1,2], None, 1)
>>> type(a)
<type 'avl_tree'>
>>> a
[1, 2]
\end{Verbatim}
\end{funcdesc}

% 
% Pickling
% 
\begin{funcdesc}{dump}{tree,pickler}
	Convenience function to pickle a tree, as in
	\code{tree.dump(pickler)}.  First we pickle the size of the
	tree as a \code{PyInt} object, then its compare function, then
	each item in order by \code{pickler.dump()}.  The \module{cPickle}
	module has no exported API. The advantage of visiting the tree in
	inorder is that no comparison is necessary at unpickling time.
	
	Note that \var{pickler} is not type-checked (the function
	only checks for the existence of a callable `\var{dump}' attribute).
\end{funcdesc}
\begin{funcdesc}{load}{unpickler}
	Convenience function to unpickle a tree which was pickled with the
	simple method applied by \function{avl.dump} (see above).
	
	Note that \var{unpickler} is not type-checked (the
	function only checks for the existence of a callable `\var{load}'
	attribute).
\end{funcdesc}
\begin{funcdesc}{from_iter}{%
	iter\optional{, len},\optional{, compare}%
	}
	Load a tree from \var{iter} if the sequence is in sorted order
	with respect to some \var{compare} function.  This can't be
	reproduced in Python since it relies on \cfunction{avl_xload} in
	\code{avl.c} (which proceeds recursively like
	\cfunction{avl_slice} and has to know the items count in advance).
	If no \var{len} is specified, it will be read by the first call to
	\code{iter.next()}.  It can be specified as an \code{intobject} or
	a \code{longobject}.  If no \var{compare} function is specified,
	\code{None} is assumed.
	
	A \exception{StopIteration} exception occurs if \var{len} is too
	large.
	
	Note that \var{iter} is not type-checked (the function only checks
	for the existence of a callable `\var{next}' attribute).
\end{funcdesc}

\begin{excdesc}{avl.Error}
Exception raised when an operation fails for some \module{avl}
specific reason.  The exception argument is a string describing the
reason for failure or just where it occurred.
\end{excdesc}

%%%%%%%%%%%%%%%%%%%%%%%%
%                      %
%   AVL-tree objects   %
%                      %
%%%%%%%%%%%%%%%%%%%%%%%%
\subsection{`avl_tree' objects}
\label{avl-tree-objects}

\textsc{AVL} objects, as returned by \function{new()} above, have the
following functions~
\note{functions which do comparisons to do their work will raise
an exception as soon as some comparison procedure fails.}

First, here is a function that is included by defining the  
\constant{HAVE_AVL_VERIFY} compile flag~:
\begin{methoddesc}[avl_tree]{verify}{}
	Verify internal AVL tree structure, including \emph{ordering}.
	Return \code{1} if tree is valid, \code{0} otherwise.
\end{methoddesc}

\begin{methoddesc}[avl_tree]{lookup}{item}
	Return new reference to an item in the tree that compares equal to passed 
	\var{item}. Raise a \exception{LookupError} exception if \var{item} is
	not contained in the tree. 
\end{methoddesc}

\begin{methoddesc}[avl_tree]{insert}{item\optional{, index}}
	Insert \var{item} in the tree.  If no \var{index} is specified,
	\var{item} is inserted based on its rank with respect to the
	built-in Python compare function \cfunction{PyObject_Compare()},
	or a possible user-supplied one.  Indeed items of any type can be
	inserted as long as they are comparable to one another.  If some
	error occurs during a compare operation, the tree remains
	unchanged and an exception is raised.  Otherwise insertion is
	carried out whether or not an item comparing equal to \var{item}
	is already present in the tree.

	Use the call \samp{t.insert(o,j)} to insert \code{o} in front of
	index \code{j} in \code{t} regardless of order, if it's what you
	really want (an \exception{IndexError} exception is raised if
	\code{j} is out of range).  \note{order may be broken as a result. 
	See also the \method{span()} method.}
\end{methoddesc}

\begin{methoddesc}[avl_tree]{append}{item}
	Shortcut to append \var{item} to tree regardless of order.
\end{methoddesc}

\begin{methoddesc}[avl_tree]{remove}{item}
	Remove \var{item} from the tree. Nothing is done if \var{item} is not found.
\end{methoddesc}

\begin{methoddesc}[avl_tree]{clear}{}
	Make the tree logically empty in one sweep.
\end{methoddesc}

\begin{methoddesc}[avl_tree]{remove_at}{index}
	Remove item in front of specified \var{index}.
	An \code{IndexError} exception is raised if \var{index} is out of range.
\end{methoddesc}

\begin{methoddesc}[avl_tree]{has_key}{item}
	Return \code{1} if \var{item} is in tree, \code{0} otherwise.
\end{methoddesc}

\begin{methoddesc}[avl_tree]{index}{item}
	\samp{t.index(item)} returns smallest index \code{j} such that
	\code{t[j] == \var{item}}, or \code{-1} if \var{item} is not in \code{t}.
\end{methoddesc}

\begin{methoddesc}[avl_tree]{concat}{arg}
	In-place concatenation~: 
	append tree \var{arg} to the tree, regardless of order.
\end{methoddesc}

\begin{methoddesc}[avl_tree]{min}{}
	Return smallest item in tree, except if it's empty.
\end{methoddesc}

\begin{methoddesc}[avl_tree]{max}{}
	Return greatest item in tree, except if it's empty.
\end{methoddesc}

\begin{methoddesc}[avl_tree]{at_least}{item}
	Return smallest item that compares greater than or equal to \var{item} if 
	any, or raise a \exception{ValueError} exception.
\end{methoddesc}

\begin{methoddesc}[avl_tree]{at_most}{item}
	Return greatest item that compares less than or equal to \var{item} if 
	any, or raise a \exception{ValueError} exception.
\end{methoddesc}

% We might change that method's protocol
\begin{methoddesc}[avl_tree]{span}{o1\optional{, o2}}
	\samp{t.span(o1)} returns a pair of indices \code{(i,j)} such that 
	\code{t[i:j]} is the longest slice that spans \code{o1} if it's in \code{t},
	otherwise \code{i==j} in which \samp{t.insert(o1,j)} puts \code{o1} 
	where it should be with no need to redo comparisons.
	
	\samp{t.span(o1,o2)} returns \code{(i,j)} such that 
	\code{t[i:j]} is the longest slice that spans \code{o1,o2}.
	For example (see below for slice support),
\begin{Verbatim}
>>> a= avl.new(map (lambda x: random.randint(0,1000), range(10)))
>>> a
[28, 66, 82, 95, 109, 114, 268, 335, 761, 851]
>>> a.span(100,500)
(4, 8)
>>> a[4:8]
[109, 114, 268, 335]
>>> a.span(500,100)
(4, 8)
>>> a.span(900,950)
(10, 10)
>>> len(a)
10
>>> a.span(a[0],a[-1])
(0, 10)
>>> a.span(95,109)
(3, 5)
\end{Verbatim}
\end{methoddesc}

% Pickle method
\begin{methoddesc}[avl_tree]{dump}{pickler}
	See \function{avl.dump} module function.
\end{methoddesc}

% Get an iterator
\begin{methoddesc}[avl_tree]{iter}{\optional{pre_or_post}}
	Return a new iterator over the items in this tree, either in
	pre-position if \var{pre_or_post} is zero/false (the default), or in
	post-position.
\end{methoddesc}

%%%%%%%%%%%%%%%%%%%%%%%%%
%                       %
%   Sequence protocol   %
%                       %
%%%%%%%%%%%%%%%%%%%%%%%%%
There is support for the \textbf{sequence} protocol.

\begin{itemize}
	\item  \samp{len(t)} returns the size of tree \code{t}.

	\item  \samp{a+b} returns a new tree object resulting from the concatenation 
	of trees \code{`a'} and \code{`b'}. This is done regardless of order.

	\item  The \code{repeat} operation is undefined.
	
	\item  \code{`t[j]'} returns a new reference to the item whose inorder index 
	in \code{t} is \code{j} (starting from \code{0}).
	The usual defaults apply~: \code{j} should be in range 
	\code{[-len(t):len(t)]}, otherwise
	an \exception{IndexError} exception is raised.
	
	\item  A \strong{slice} of a tree can be obtained as a new tree, with the 
	usual defaults.
	Thus \samp{b = a[:]} is equivalent to \samp{b = avl.new(a)}.
	No exception.
	
	\item  Item and slice assignments are undefined.
	
	\item  \samp{o in t} is equivalent to \samp{t.has_key(o)}.
	
	\item To perform in-place concatenation please call \samp{a += b} 
	or \samp{a.concat(b)}.
\end{itemize}

%%%%%%%%%%%%%%%%%%%%%%%%%
%                       %
%   Iterator protocol   %
%                       %
%%%%%%%%%%%%%%%%%%%%%%%%%
\subsection{Support for the \textbf{iterator} protocol: `avl_tree_iterator' objects}
\label{iterator}

\begin{seealso}
\seepep{234}{Iterators}{New protocol in Python 2.2}
\end{seealso}

The \module{avl} module implements the iterator protocol. 
Objects of type \code{`avl_tree_iterator'} are implicitly called upon 
if a \code{for} loop is used to iterate over a tree,
like in \samp{for o in t: print o}. This is more 
efficient than retrieving the $i^{\mathrm{th}}$ item for all $i$.

There are also a couple of explicit methods.

To get a new iterator (in pre-position) for some tree \code{t}, say
\samp{iter(t)} in Python~:
\begin{Verbatim}
	>>> j = iter(t)
	>>> type(j)
	<type 'avl_tree_iterator'>
\end{Verbatim}
Note that it increments the tree object's refcount.

To obtain an iterator in either pre- or post-position,
use the \function{iter} instance method, see above.
The call \samp{t.iter()} is equivalent to \samp{iter(t)}.

Here are the methods~:

\begin{methoddesc}[avl_tree_iterator]{next}{}
	Return new reference to next item if there is one, or raise
	a  \code{StopIteration} exception. This is part of the protocol.
\end{methoddesc}

\begin{methoddesc}[avl_tree_iterator]{prev}{}
	Symmetrically, return new reference to previous item.
	This is not part of the protocol.
\end{methoddesc}

\begin{methoddesc}[avl_tree_iterator]{index}{}
	Return inorder index of current item in iteration, 
	or \code{-1} if it's in `pre-position', i.e. right before the
	first item, or tree's \code{len} if it's in `post-position'.
\end{methoddesc}

Since the tree implementation maintains a parent pointer for each
node, any iterator remains able to proceed if insertions are done
while it's in use, or if any item \emph{\textbf{other than the current
one}} is deleted via the \method{remove()} tree object method.  The
module provides another iterator method~:

\begin{methoddesc}[avl_tree_iterator]{remove}{}
	Remove current item in iteration, or raise an
	\exception{avl.Error} exception.  Current iterator position is set
	to next position or in post-position. This is not part of the protocol.
\end{methoddesc}

%%%%%%%%%%%%%%%
%             %
%   Example   %
%             %
%%%%%%%%%%%%%%%
\subsection{Example \label{avl-example}}

The following example demonstrates usage of the \module{avl} module.

\begin{Verbatim}[fontsize=\small,baselinestretch=.9,commentchar=!]
Python 2.3 (#1, Sep 13 2003, 00:49:11) 
[GCC 3.3 20030304 (Apple Computer, Inc. build 1495)] on darwin
Type "help", "copyright", "credits" or "license" for more information.
>>> import random
>>> import avl

# Create new avl_tree:
>>> t = avl.new()
>>> type(t)
<type 'avl_tree'>

# Insert new items:
>>> for x in range(20): t.insert(random.randint(0, 100))
... 
# verify() --> 1 if tree is valid, 0 otherwise
>>> t.verify()
1
>>> t
[2, 5, 9, 18, 24, 25, 29, 42, 45, 51, 58, 58, 60, 64, 66, 80, 85, 87, 92, 99]

# Lookup by index with usual syntax:
>>> t[0], t.min()
(2, 2)
>>> t[-1], t.max()
(99, 99)
>>> t[4], t[-16]
(24, 24)

# Create a tree from list of items:
>>> list = [3,8,3,5,1,2,8,7]
>>> u = avl.new(list)
# As a result, list is sorted ; this is an _unstable_ sort
>>> u
[1, 2, 3, 3, 5, 7, 8, 8]
# A second optional argument to new() is a Python compare function:
>>> u= avl.new(list, None)
>>> u
[1, 2, 3, 3, 5, 7, 8, 8]
# Same as above except duplicates are removed:
>>> u = avl.new(list, None, 1)
>>> u
[1, 2, 3, 5, 7, 8]

# Lookup by key:
>>> t.lookup(42)
42
# Failure:
>>> t.lookup(36)
Traceback (most recent call last):
  File "<stdin>", line 1, in ?
LookupError: 36

# Smallest index of item, or -1:
>>> t.index(58)
10
>>> i,j = t.span(58); i,j
(10, 12)
>>> t[10:12]
[58, 58]
>>> t.span(36)
(7, 7)
# Put it where it should be without comparing:
>>> t.insert(36,7); t.verify()
1
>>> t.index(36)
7

# Slices with usual defaults:
>>> u = t[:10]
>>> type(u)
<type 'avl_tree'>
>>> u
[2, 5, 9, 18, 24, 25, 29, 36, 42, 45]
>>> 42 in u
True
# otherwise put:
>>> u.has_key(42)
1

# new() accepts a tree as argument
# this is equivalent to `a = u[:]':
>>> a = avl.new(u)
>>> a.verify()
1
>>> b = avl.new(range(60,70))

### Concatenation with usual syntax:
>>> c = a + b
>>> c.verify()
1
>>> c
[2, 5, 9, 18, 24, 25, 29, 36, 42, 45, 60, 61, 62, 63, 64, 65, 66, 67, 68, 69]

# Concatenation is done regardless of order:
>>> avl.new([5,1,2]) + avl.new([2,8,6])
[1, 2, 5, 2, 6, 8]

# In-place concatenation:
>>> a.concat(b)
>>> a
[2, 5, 9, 18, 24, 25, 29, 36, 42, 45, 60, 61, 62, 63, 64, 65, 66, 67, 68, 69]

### ITERATION:
>>> n = 0
>>> for o in u:
... 	print n, ":", o
... 	n += 1
... 
0 : 2
1 : 5
2 : 9
3 : 18
4 : 24
5 : 25
6 : 29
7 : 36
8 : 42
9 : 45

### Explicit iteration:
>>> j = iter(u)
>>> type(j)
<type 'avl_tree_iterator'>
>>> j.next()
2
>>> j.next()
5
>>> u
[2, 5, 9, 18, 24, 25, 29, 36, 42, 45]
>>> j.index()
1
>>> u[2]
9
>>> u.insert(11)
>>> j.next()
9
>>> j.next()
11
>>> j.cur()
11
>>> j.remove()
>>> j.next()
18
>>> 11 in u
False
>>> u.verify()
1
# Get previous item:
>>> j.prev()
9

### Removing items:

# Remove u[3]:
>>> u.remove_at(3)
>>> u.verify()
1
# Remove by key:
>>> u.remove(24)
>>> u.remove(36)
>>> u
[2, 5, 9, 25, 29, 42, 45]
>>> u.verify()
1
# Nothing is done if item not found:
>>> u.remove(20)
>>> u
[2, 5, 9, 25, 29, 42, 45]

### at_least() and at_most(), with exceptions:
>>> u.at_least(30)
42
>>> u.at_most(30)
29
>>> u.at_least(50)
Traceback (most recent call last):
  File "<input>", line 1, in ?
ValueError: 50

! The span() protocol may be subject to change
>>> u.span(1,9)
(0, 3)
>>> u[0:3]
[2, 5, 9]
>>> u.span(2,8)
(0, 2)
>>> u.span(30,35)
(5, 5)
>>> u[5]
42
>>> u.span(25,44)
(3, 6)
>>> u[3:6]
[25, 29, 42]
>>> u.span(24,44)
(3, 6)
>>> u.span(24, 45)
(3, 7)
>>> u[3:7]
[25, 29, 42, 45]
>>> u.span(26,45)
(4, 7)
>>> 
\end{Verbatim}
% 
%%%%%%%%%%%%%%%%%%
%                %
%   References   %
%                %
%%%%%%%%%%%%%%%%%%
\clearpage
\begin{thebibliography}{9}\addcontentsline{toc}{section}{References}
	
	\bibitem{Rushing} Sam Rushing's own iterative C implementation, on
	which this one is based, is to be found at
	\url{http://www.python.org/ftp/python/contrib-09-Dec-1999/DataStructures/avl.README}
	
	\bibitem{Pfaff} Good ideas came from browsing 
	Ben Pfaff's GNU libavl home at \url{http://adtinfo.org/}
	
	\bibitem{AVL} Adel'son-Vel'ski\u{\i} (G.M.) and Landis (E.M.) ---
	{\itshape An algorithm for the organization of information.} Soviet
	Mathematics Doklady, vol.~3, 1962, pp.  1259--1263
\end{thebibliography}

\input{pyavl.ind}\addcontentsline{toc}{section}{Index}
% \input{modpyavl.ind}
\end{document}
\addcontentsline{toc}{section}{Index}
% \input{modpyavl.ind}
\end{document}
\addcontentsline{toc}{section}{Index}
% \input{modpyavl.ind}
\end{document}
\addcontentsline{toc}{section}{Index}
% \input{modpyavl.ind}
\end{document}
